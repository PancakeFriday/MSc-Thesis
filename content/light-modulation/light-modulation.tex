\chapter{Theory of light modulation}

The concepts of our ultracold atom experiment were highlighted in \mbox{Chapter \ref{ch:motivation}}, most importantly, coherent laser systems are used to modify the state of potassium atoms, which allows to induce interactions between the atoms and image them onto a camera. Having control over the electromagnetic waves by means of electronically controlled light modulators therefore gives direct control over the atomic system at each step of an experimental run.
Two important components are highlighted in the following, these are \acp{eom} and \acp{aom}.

\section{Electro-optical modulators}
\label{sec:eom}

\todo{ref fundamentals of photonics somewhere}

Materials can change their refractive index by being exposed to an electric field. In \acp{eom}, this is generally a non-linear crystal connected to two electrodes. There are two prominent electro-optical effects which need to be distinguished. If the refractive index changes linearly with the electric field, the effect is called Pockels effect and the \ac{eom} is called a pockels cell. However if it changes with the square of the electric field, the effect is called Kerr effect. For this thesis, two Pockels cells from Leysop Ltd. were studied and therefore we will only discuss the pockels effect after going through some basic polarization theory.

\subsection{Polarization}

\label{sec:pol}

Amongst frequency and amplitude, there is another parameter that can generally be affected in electromagnetic waves, which is the polarization. The polarization is the orientation of the wave in space. In general, a wave travelling along the z-axis can be oriented somewhere in the x-y plane. Therefore, writing the electromagnetic field in this basis takes the following form:

\begin{equation}
	E(\mathbf{x}, t) = E_x cos\left(kx - wt + \phi_x\right) \mathbf{e_x} + E_y cos\left(ky - wt + \phi_y\right) \mathbf{e_y} .
\end{equation}

Here, $k$ and $w$ refer to the wave number and frequency respectively.
Depending on the amplitudes $E_x$ and $E_y$ and the phases $\phi_x$ and $\phi_y$, the wave can be in different polarization states. If it is not possible to write the wave in this basis, the light is unpolarized. Otherwise, it is \textbf{linear}, when either one of the amplitudes $E_x$ or $E_y$ is zero or when the phase difference $\Delta \phi = \phi_x - \phi_y$ evaluates to 0 or $\pi$. It is \textbf{circular}, when the phase difference $\Delta \phi = \pm \pi/2$ and the amplitudes are the same, $E_x = E_y$. In any other case, the wave is \textbf{elliptically} polarized.

From this point, we can see how to create a device that modifies the polarization. We can do this by retarding one axis stronger than another. Given a material with two refractive indices $n_x$ and $n_y$ along the axes $x$ and $y$, we then get the phase shifts:

\begin{align}
	\phi_x(z) = k_0 n_x z \\
	\phi_y(z) = k_0 n_y z
\end{align}
\label{eq:pol,phases}

where $k_0$ is the free space wave vector of the light. Then a device that retards the phase difference $\Delta \phi$ by $\pi/2$, which is a quarter of the wavelength, can change linearly polarized light to circularly polarized light (or vice-versa) and is therefore called a $\lambda / 4$ waveplate. Similarily, if the phase difference is changed by $\Delta \phi = \pi$, or a half wavelength, then we can turn linear polarization around a given axis or change the orientation of circularly polarized light. This is then called a $\lambda / 2$ waveplate.

\subsection{Pockels effect}

The derivation of the pockels effect is fairly straightforward and as such, we follow the argumentation from the book Fundamentals of Photonics \todo{ref}.

Knowing that the refractive index $n$ of the crystal in the pockels cell depends on the electric field, we can write it as $n(E)$. Applying a taylor expansion, we get the following expression:

\begin{equation}
	n(E) = n_0 + \frac{dn}{dE} E + \mathcal{O}(E^2)
\end{equation}

As was motivated before in Section \ref{sec:eom}, the pockels effect is the linear dependence of the refractive index to the electric field, therefore higher orders are neglected. To write the expression in terms of a more physically relevant quantity, we need to look at the electric impermeability $\eta$ and its error $\Delta \eta$:

\begin{align}
	\eta & = \frac{1}{n_0^2} \\
	\Delta \eta = \frac{d \eta}{dn_0} \Delta n & = -\frac{2}{n_0^3} \frac{dn}{dE} E = \mathfrak{r} E .
\end{align}
\label{eq:pockel,refr}

This results in the quantity $\mathfrak{r} = -\frac{2}{n_0^3} \frac{dn}{dE}$, which is called the Pockels coefficient given in units of $\SI{}{\meter\per\volt}$. It can be measured by evaluating the refractive index of the material:

\begin{equation}
	n(E) = n_0 - \frac{1}{2} \mathfrak{r} n_0^3 E .
\end{equation}

The pockels cells in this application act as dynamic wave retarders, therefore with the results from section \ref{sec:pol}, we can tune the phase difference $\Delta \phi = \phi_x - \phi_y$ along the axes $x$ and $y$ by applying an electric field.

%The phase shift due to the pockels effect is then calculated via \ref{eq:pol,phases} and \ref{eq:pockel,refr}:
We can see the effect on the phase difference by combining \ref{eq:pol,phases} and \ref{eq:pockel,refr}:

\begin{align}
	\phi & = k_0 L n \\
		 & = k_0 L n_0 - \frac{k_0}{2} L \mathfrak{r} n_0^3 E \\
		 & = \phi_0 - \frac{k_0}{2} L \mathfrak{r} n_0^3 E \\
		 & = \phi_0 - \frac{\pi}{\lambda_0} L \mathfrak{r} n_0^3 E \\
\end{align}

where the relation $k_0 = 2 \pi / \lambda_0$ of the wave number was used.

It is now instructive to calculate the phase difference, which gives information about the change in polarization. The relations for the refractive indices in the two axis basis are labeled as:

\begin{align}
	n_x(E) = n_{0,x} - \frac{1}{2} \mathfrak{r}_x n_{0,x}^3 E \\
	n_y(E) = n_{0,y} - \frac{1}{2} \mathfrak{r}_y n_{0,y}^3 E
\end{align}

and then the phase difference becomes:

\begin{align}
	\Delta \phi & = \phi_{0,x} - \phi_{0,y} - \frac{\pi}{\lambda_0} E L \left(\mathfrak{r}_x n_x^3 - \mathfrak{r}_y n_y^3\right) \\
	\Delta \phi & = \Delta \phi_{0} - \frac{\pi}{\lambda_0} E L \left(\mathfrak{r}_x n_x^3 - \mathfrak{r}_y n_y^3\right) .
\end{align}

The electric field is generated by applying a voltage $V$ to two electrodes that are separated by a distance $d$ and therefore $E = V/d$. We can then define a half-wave voltage $V_\pi$:

\begin{align}
	V_\pi = \frac{d}{L} \frac{\lambda_0}{\mathfrak{r}_x n_x^3 - \mathfrak{r}_y n_y^3}
\end{align}

and the phase difference becomes:

\begin{align}
	\Delta \phi = \Delta \phi_0 - \pi \frac{V}{V_\pi} .
\end{align}

With this it is clear, that applying the voltage $V_\pi$, the pockels cell will act as a lambda-half waveplate.

\begin{figure}[h]
\centering{
%LaTeX with PSTricks extensions
%%Creator: Inkscape 1.0 (4035a4fb49, 2020-05-01)
%%Please note this file requires PSTricks extensions
\psset{xunit=.5pt,yunit=.5pt,runit=.5pt}
\begin{pspicture}(793.7007874,1122.51968504)
{
\newrgbcolor{curcolor}{0.15686275 0.35294119 0.47450981}
\pscustom[linestyle=none,fillstyle=solid,fillcolor=curcolor]
{
\newpath
\moveto(222.76641078,611.48306772)
\curveto(225.57573528,612.53917036)(228.38260384,613.59386929)(231.00784252,614.54682283)
\curveto(233.63061959,615.49841945)(236.06688491,616.34548667)(238.16685165,617.00415296)
\curveto(240.26685581,617.66284264)(242.02810847,618.13175109)(243.37214135,618.36689547)
\curveto(244.71859087,618.60342016)(245.64045808,618.60203985)(246.14502917,618.36689547)
\curveto(246.65209323,618.13317819)(246.73695383,617.66284264)(246.48559106,617.00415296)
\curveto(246.234192,616.34548667)(245.6465692,615.49841945)(244.87237002,614.54682283)
\curveto(244.09821014,613.5952496)(243.13747238,612.53917036)(242.17677279,611.48306772)
}
}
{
\newrgbcolor{curcolor}{0 0 0}
\pscustom[linewidth=1.13385751,linecolor=curcolor]
{
\newpath
\moveto(222.76641078,611.48306772)
\curveto(225.57573528,612.53917036)(228.38260384,613.59386929)(231.00784252,614.54682283)
\curveto(233.63061959,615.49841945)(236.06688491,616.34548667)(238.16685165,617.00415296)
\curveto(240.26685581,617.66284264)(242.02810847,618.13175109)(243.37214135,618.36689547)
\curveto(244.71859087,618.60342016)(245.64045808,618.60203985)(246.14502917,618.36689547)
\curveto(246.65209323,618.13317819)(246.73695383,617.66284264)(246.48559106,617.00415296)
\curveto(246.234192,616.34548667)(245.6465692,615.49841945)(244.87237002,614.54682283)
\curveto(244.09821014,613.5952496)(243.13747238,612.53917036)(242.17677279,611.48306772)
}
}
{
\newrgbcolor{curcolor}{0.3882353 0.3882353 0.39607844}
\pscustom[linestyle=none,fillstyle=solid,fillcolor=curcolor,opacity=0.435294]
{
\newpath
\moveto(222.76815345,611.48238926)
\curveto(223.69245732,613.2791347)(224.61675969,615.07354063)(225.54106205,616.69482264)
\curveto(226.46535685,618.31376515)(227.38967811,619.75490472)(228.31396157,620.8755311)
\curveto(229.23828283,621.99615749)(230.1625663,622.79393118)(231.08688756,623.19398778)
\curveto(232.01117102,623.5963839)(232.93549228,623.59404439)(233.85977575,623.19398778)
\curveto(234.78409701,622.79627069)(235.70838047,621.99615749)(236.63270173,620.8755311)
\curveto(237.5569852,619.75490472)(238.48130646,618.31376515)(239.40558992,616.69482264)
\curveto(240.32991118,615.07588014)(241.25419465,613.2791347)(242.17851591,611.48238926)
}
}
{
\newrgbcolor{curcolor}{0 0 0}
\pscustom[linewidth=1.13385751,linecolor=curcolor]
{
\newpath
\moveto(222.76815345,611.48238926)
\curveto(223.69245732,613.2791347)(224.61675969,615.07354063)(225.54106205,616.69482264)
\curveto(226.46535685,618.31376515)(227.38967811,619.75490472)(228.31396157,620.8755311)
\curveto(229.23828283,621.99615749)(230.1625663,622.79393118)(231.08688756,623.19398778)
\curveto(232.01117102,623.5963839)(232.93549228,623.59404439)(233.85977575,623.19398778)
\curveto(234.78409701,622.79627069)(235.70838047,621.99615749)(236.63270173,620.8755311)
\curveto(237.5569852,619.75490472)(238.48130646,618.31376515)(239.40558992,616.69482264)
\curveto(240.32991118,615.07588014)(241.25419465,613.2791347)(242.17851591,611.48238926)
}
}
{
\newrgbcolor{curcolor}{0.3882353 0.3882353 0.39607844}
\pscustom[linestyle=none,fillstyle=solid,fillcolor=curcolor,opacity=0.435294]
{
\newpath
\moveto(242.17678035,611.48571137)
\curveto(243.10108422,609.68896593)(244.02538658,607.89456)(244.94968894,606.27327799)
\curveto(245.87398375,604.65433548)(246.79830539,603.21319591)(247.72258885,602.09256952)
\curveto(248.64690973,600.97194314)(249.57119357,600.17416945)(250.49551483,599.77411285)
\curveto(251.4197983,599.37171673)(252.34411994,599.37405624)(253.26840265,599.77411285)
\curveto(254.19272391,600.17182994)(255.11700775,600.97194314)(256.04132863,602.09256952)
\curveto(256.96561285,603.21319591)(257.88993373,604.65433548)(258.81421682,606.27327799)
\curveto(259.73853921,607.89222049)(260.66283591,609.68896593)(261.58715717,611.48571137)
}
}
{
\newrgbcolor{curcolor}{0 0 0}
\pscustom[linewidth=1.13385751,linecolor=curcolor]
{
\newpath
\moveto(242.17678035,611.48571137)
\curveto(243.10108422,609.68896593)(244.02538658,607.89456)(244.94968894,606.27327799)
\curveto(245.87398375,604.65433548)(246.79830539,603.21319591)(247.72258885,602.09256952)
\curveto(248.64690973,600.97194314)(249.57119357,600.17416945)(250.49551483,599.77411285)
\curveto(251.4197983,599.37171673)(252.34411994,599.37405624)(253.26840265,599.77411285)
\curveto(254.19272391,600.17182994)(255.11700775,600.97194314)(256.04132863,602.09256952)
\curveto(256.96561285,603.21319591)(257.88993373,604.65433548)(258.81421682,606.27327799)
\curveto(259.73853921,607.89222049)(260.66283591,609.68896593)(261.58715717,611.48571137)
}
}
{
\newrgbcolor{curcolor}{0.15686275 0.35294119 0.47450981}
\pscustom[linestyle=none,fillstyle=solid,fillcolor=curcolor]
{
\newpath
\moveto(242.17852346,611.48503291)
\curveto(241.21780687,610.42893027)(240.2595428,609.37423134)(239.48290847,608.4212778)
\curveto(238.70872101,607.46968118)(238.12109934,606.62261396)(237.86969915,605.96394767)
\curveto(237.61833789,605.30525799)(237.70565254,604.83634954)(238.21026104,604.60120516)
\curveto(238.71237921,604.36468047)(239.63915452,604.36606078)(240.98314923,604.60120516)
\curveto(242.32472731,604.83492244)(244.08843439,605.30525799)(246.18843893,605.96394767)
\curveto(248.28840605,606.62261396)(250.72467061,607.46968118)(253.34743597,608.4212778)
\curveto(255.97024101,609.37285103)(258.77955742,610.42893027)(261.58889953,611.48503291)
}
}
{
\newrgbcolor{curcolor}{0 0 0}
\pscustom[linewidth=1.13385751,linecolor=curcolor]
{
\newpath
\moveto(242.17852346,611.48503291)
\curveto(241.21780687,610.42893027)(240.2595428,609.37423134)(239.48290847,608.4212778)
\curveto(238.70872101,607.46968118)(238.12109934,606.62261396)(237.86969915,605.96394767)
\curveto(237.61833789,605.30525799)(237.70565254,604.83634954)(238.21026104,604.60120516)
\curveto(238.71237921,604.36468047)(239.63915452,604.36606078)(240.98314923,604.60120516)
\curveto(242.32472731,604.83492244)(244.08843439,605.30525799)(246.18843893,605.96394767)
\curveto(248.28840605,606.62261396)(250.72467061,607.46968118)(253.34743597,608.4212778)
\curveto(255.97024101,609.37285103)(258.77955742,610.42893027)(261.58889953,611.48503291)
}
}
{
\newrgbcolor{curcolor}{0.15686275 0.35294119 0.47450981}
\pscustom[linestyle=none,fillstyle=solid,fillcolor=curcolor]
{
\newpath
\moveto(249.91333422,611.4830587)
\curveto(251.70816689,612.54697914)(253.5014305,613.60948548)(255.17865367,614.56949327)
\curveto(256.85430416,615.52813409)(258.41079527,616.38147172)(259.75243057,617.04501379)
\curveto(261.09408977,617.70857943)(262.21932604,618.18095897)(263.07800724,618.41784401)
\curveto(263.93823238,618.65611958)(264.52719866,618.65472905)(264.84956109,618.41784401)
\curveto(265.17351624,618.18239664)(265.22773232,617.70857943)(265.06714065,617.04501379)
\curveto(264.9065258,616.38147172)(264.53110295,615.52813409)(264.03647943,614.56949327)
\curveto(263.54188102,613.61087601)(262.92808096,612.54697914)(262.3143053,611.4830587)
}
}
{
\newrgbcolor{curcolor}{0 0 0}
\pscustom[linewidth=0.90964309,linecolor=curcolor]
{
\newpath
\moveto(249.91333422,611.4830587)
\curveto(251.70816689,612.54697914)(253.5014305,613.60948548)(255.17865367,614.56949327)
\curveto(256.85430416,615.52813409)(258.41079527,616.38147172)(259.75243057,617.04501379)
\curveto(261.09408977,617.70857943)(262.21932604,618.18095897)(263.07800724,618.41784401)
\curveto(263.93823238,618.65611958)(264.52719866,618.65472905)(264.84956109,618.41784401)
\curveto(265.17351624,618.18239664)(265.22773232,617.70857943)(265.06714065,617.04501379)
\curveto(264.9065258,616.38147172)(264.53110295,615.52813409)(264.03647943,614.56949327)
\curveto(263.54188102,613.61087601)(262.92808096,612.54697914)(262.3143053,611.4830587)
}
}
{
\newrgbcolor{curcolor}{0.3882353 0.3882353 0.39607844}
\pscustom[linestyle=none,fillstyle=solid,fillcolor=curcolor,opacity=0.435294]
{
\newpath
\moveto(249.91444758,611.48237522)
\curveto(250.50497061,613.29242107)(251.09549266,615.10011009)(251.68601472,616.73339366)
\curveto(252.27653194,618.36432039)(252.86706607,619.816128)(253.45757605,620.94504983)
\curveto(254.04811018,622.07397166)(254.63862016,622.87765087)(255.22915429,623.28066889)
\curveto(255.81966427,623.68604375)(256.4101984,623.68368692)(257.00070838,623.28066889)
\curveto(257.59124251,622.8800077)(258.18175249,622.07397166)(258.77228662,620.94504983)
\curveto(259.3627966,619.816128)(259.95333073,618.36432039)(260.54384071,616.73339366)
\curveto(261.13437484,615.10246692)(261.72488482,613.29242107)(262.31541895,611.48237522)
}
}
{
\newrgbcolor{curcolor}{0 0 0}
\pscustom[linewidth=0.90964309,linecolor=curcolor]
{
\newpath
\moveto(249.91444758,611.48237522)
\curveto(250.50497061,613.29242107)(251.09549266,615.10011009)(251.68601472,616.73339366)
\curveto(252.27653194,618.36432039)(252.86706607,619.816128)(253.45757605,620.94504983)
\curveto(254.04811018,622.07397166)(254.63862016,622.87765087)(255.22915429,623.28066889)
\curveto(255.81966427,623.68604375)(256.4101984,623.68368692)(257.00070838,623.28066889)
\curveto(257.59124251,622.8800077)(258.18175249,622.07397166)(258.77228662,620.94504983)
\curveto(259.3627966,619.816128)(259.95333073,618.36432039)(260.54384071,616.73339366)
\curveto(261.13437484,615.10246692)(261.72488482,613.29242107)(262.31541895,611.48237522)
}
}
{
\newrgbcolor{curcolor}{0.3882353 0.3882353 0.39607844}
\pscustom[linestyle=none,fillstyle=solid,fillcolor=curcolor,opacity=0.435294]
{
\newpath
\moveto(262.31431013,611.48572192)
\curveto(262.90483315,609.67567606)(263.49535521,607.86798704)(264.08587726,606.23470348)
\curveto(264.67639449,604.60377674)(265.26692886,603.15196913)(265.85743884,602.0230473)
\curveto(266.44797273,600.89412548)(267.03848295,600.09044626)(267.62901708,599.68742824)
\curveto(268.21952706,599.28205339)(268.81006143,599.28441022)(269.40057093,599.68742824)
\curveto(269.99110506,600.08808943)(270.58161528,600.89412548)(271.17214917,602.0230473)
\curveto(271.76265963,603.15196913)(272.35319352,604.60377674)(272.94370326,606.23470348)
\curveto(273.53423811,607.86563021)(274.12475654,609.67567606)(274.71529067,611.48572192)
}
}
{
\newrgbcolor{curcolor}{0 0 0}
\pscustom[linewidth=0.90964309,linecolor=curcolor]
{
\newpath
\moveto(262.31431013,611.48572192)
\curveto(262.90483315,609.67567606)(263.49535521,607.86798704)(264.08587726,606.23470348)
\curveto(264.67639449,604.60377674)(265.26692886,603.15196913)(265.85743884,602.0230473)
\curveto(266.44797273,600.89412548)(267.03848295,600.09044626)(267.62901708,599.68742824)
\curveto(268.21952706,599.28205339)(268.81006143,599.28441022)(269.40057093,599.68742824)
\curveto(269.99110506,600.08808943)(270.58161528,600.89412548)(271.17214917,602.0230473)
\curveto(271.76265963,603.15196913)(272.35319352,604.60377674)(272.94370326,606.23470348)
\curveto(273.53423811,607.86563021)(274.12475654,609.67567606)(274.71529067,611.48572192)
}
}
{
\newrgbcolor{curcolor}{0.15686275 0.35294119 0.47450981}
\pscustom[linestyle=none,fillstyle=solid,fillcolor=curcolor]
{
\newpath
\moveto(262.31542378,611.48503844)
\curveto(261.70163725,610.42111799)(261.0894176,609.35861165)(260.5932383,608.39860387)
\curveto(260.09862226,607.43996305)(259.72320014,606.58662541)(259.56258456,605.92308334)
\curveto(259.40199386,605.2595177)(259.4577778,604.78713816)(259.78016413,604.55025312)
\curveto(260.10095943,604.31197755)(260.69306141,604.31336808)(261.55171822,604.55025312)
\curveto(262.40883109,604.78570049)(263.53563545,605.2595177)(264.87729489,605.92308334)
\curveto(266.21893043,606.58662541)(267.77542105,607.43996305)(269.45106406,608.39860387)
\curveto(271.12673242,609.35722112)(272.92155992,610.42111799)(274.71640384,611.48503844)
}
}
{
\newrgbcolor{curcolor}{0 0 0}
\pscustom[linewidth=0.90964309,linecolor=curcolor]
{
\newpath
\moveto(262.31542378,611.48503844)
\curveto(261.70163725,610.42111799)(261.0894176,609.35861165)(260.5932383,608.39860387)
\curveto(260.09862226,607.43996305)(259.72320014,606.58662541)(259.56258456,605.92308334)
\curveto(259.40199386,605.2595177)(259.4577778,604.78713816)(259.78016413,604.55025312)
\curveto(260.10095943,604.31197755)(260.69306141,604.31336808)(261.55171822,604.55025312)
\curveto(262.40883109,604.78570049)(263.53563545,605.2595177)(264.87729489,605.92308334)
\curveto(266.21893043,606.58662541)(267.77542105,607.43996305)(269.45106406,608.39860387)
\curveto(271.12673242,609.35722112)(272.92155992,610.42111799)(274.71640384,611.48503844)
}
}
{
\newrgbcolor{curcolor}{0 0 0}
\pscustom[linewidth=0.99999871,linecolor=curcolor]
{
\newpath
\moveto(238.38556724,585.69845669)
\lineto(238.38556724,553.70275276)
\lineto(217.40494866,553.70275276)
}
}
{
\newrgbcolor{curcolor}{0.39607844 0.39607844 0.39607844}
\pscustom[linestyle=none,fillstyle=solid,fillcolor=curcolor]
{
\newpath
\moveto(180.99538728,644.1324261)
\curveto(192.69365252,644.1324261)(202.17697023,629.92273508)(202.17697023,612.39419291)
\curveto(202.17697023,594.86565074)(192.69365252,580.65595972)(180.99538728,580.65595972)
\curveto(169.29712204,580.65595972)(159.81380433,594.86565074)(159.81380433,612.39419291)
\curveto(159.81380433,629.92273508)(169.29712204,644.1324261)(180.99538728,644.1324261)
}
}
{
\newrgbcolor{curcolor}{0.82745099 0.14509805 0}
\pscustom[linewidth=1.13385831,linecolor=curcolor]
{
\newpath
\moveto(180.99538728,644.1324261)
\curveto(192.69365252,644.1324261)(202.17697023,629.92273508)(202.17697023,612.39419291)
\curveto(202.17697023,594.86565074)(192.69365252,580.65595972)(180.99538728,580.65595972)
\curveto(169.29712204,580.65595972)(159.81380433,594.86565074)(159.81380433,612.39419291)
\curveto(159.81380433,629.92273508)(169.29712204,644.1324261)(180.99538728,644.1324261)
}
}
{
\newrgbcolor{curcolor}{1 1 1}
\pscustom[linestyle=none,fillstyle=solid,fillcolor=curcolor]
{
\newpath
\moveto(236.99549878,644.1324261)
\curveto(248.69376402,644.1324261)(258.17708173,629.92273508)(258.17708173,612.39419291)
\curveto(258.17708173,594.86565074)(248.69376402,580.65595972)(236.99549878,580.65595972)
\curveto(225.29723354,580.65595972)(215.81391582,594.86565074)(215.81391582,612.39419291)
\curveto(215.81391582,629.92273508)(225.29723354,644.1324261)(236.99549878,644.1324261)
}
}
{
\newrgbcolor{curcolor}{0 0 0}
\pscustom[linewidth=1.13385831,linecolor=curcolor]
{
\newpath
\moveto(236.99549878,644.1324261)
\curveto(248.69376402,644.1324261)(258.17708173,629.92273508)(258.17708173,612.39419291)
\curveto(258.17708173,594.86565074)(248.69376402,580.65595972)(236.99549878,580.65595972)
\curveto(225.29723354,580.65595972)(215.81391582,594.86565074)(215.81391582,612.39419291)
\curveto(215.81391582,629.92273508)(225.29723354,644.1324261)(236.99549878,644.1324261)
}
}
{
\newrgbcolor{curcolor}{1 1 1}
\pscustom[linestyle=none,fillstyle=solid,fillcolor=curcolor]
{
\newpath
\moveto(180.18219591,580.56616063)
\lineto(236.41536756,581.08217953)
\lineto(236.93558173,644.13849449)
\lineto(180.68578016,644.07651024)
}
}
{
\newrgbcolor{curcolor}{0.78431374 0.78431374 0.78431374}
\pscustom[linestyle=none,fillstyle=solid,fillcolor=curcolor]
{
\newpath
\moveto(180.99538728,644.1324261)
\curveto(192.69365252,644.1324261)(202.17697023,629.92273508)(202.17697023,612.39419291)
\curveto(202.17697023,594.86565074)(192.69365252,580.65595972)(180.99538728,580.65595972)
\curveto(169.29712204,580.65595972)(159.81380433,594.86565074)(159.81380433,612.39419291)
\curveto(159.81380433,629.92273508)(169.29712204,644.1324261)(180.99538728,644.1324261)
}
}
{
\newrgbcolor{curcolor}{0 0 0.01176471}
\pscustom[linewidth=1.13385831,linecolor=curcolor]
{
\newpath
\moveto(180.99538728,644.1324261)
\curveto(192.69365252,644.1324261)(202.17697023,629.92273508)(202.17697023,612.39419291)
\curveto(202.17697023,594.86565074)(192.69365252,580.65595972)(180.99538728,580.65595972)
\curveto(169.29712204,580.65595972)(159.81380433,594.86565074)(159.81380433,612.39419291)
\curveto(159.81380433,629.92273508)(169.29712204,644.1324261)(180.99538728,644.1324261)
}
}
{
\newrgbcolor{curcolor}{0 0 0}
\pscustom[linewidth=1.13385831,linecolor=curcolor]
{
\newpath
\moveto(180.70112504,580.65418583)
\lineto(237.02972976,580.66174488)
}
}
{
\newrgbcolor{curcolor}{0 0 0}
\pscustom[linewidth=1.13385831,linecolor=curcolor]
{
\newpath
\moveto(180.70112504,644.12352756)
\lineto(237.02972976,644.13108661)
}
}
{
\newrgbcolor{curcolor}{1 1 1}
\pscustom[linestyle=none,fillstyle=solid,fillcolor=curcolor]
{
\newpath
\moveto(180.99538728,635.39432676)
\curveto(189.47291239,635.39432676)(196.34530435,625.09681608)(196.34530435,612.39419291)
\curveto(196.34530435,599.69156975)(189.47291239,589.39405907)(180.99538728,589.39405907)
\curveto(172.51786217,589.39405907)(165.64547021,599.69156975)(165.64547021,612.39419291)
\curveto(165.64547021,625.09681608)(172.51786217,635.39432676)(180.99538728,635.39432676)
}
}
{
\newrgbcolor{curcolor}{0 0 0.01176471}
\pscustom[linewidth=1.13385831,linecolor=curcolor]
{
\newpath
\moveto(180.99538728,635.39432676)
\curveto(189.47291239,635.39432676)(196.34530435,625.09681608)(196.34530435,612.39419291)
\curveto(196.34530435,599.69156975)(189.47291239,589.39405907)(180.99538728,589.39405907)
\curveto(172.51786217,589.39405907)(165.64547021,599.69156975)(165.64547021,612.39419291)
\curveto(165.64547021,625.09681608)(172.51786217,635.39432676)(180.99538728,635.39432676)
}
}
{
\newrgbcolor{curcolor}{0.15686275 0.35294119 0.47450981}
\pscustom[linestyle=none,fillstyle=solid,fillcolor=curcolor]
{
\newpath
\moveto(130.45089265,611.48305492)
\curveto(132.24572532,612.54697536)(134.03898893,613.6094817)(135.7162121,614.56948949)
\curveto(137.39186259,615.52813031)(138.94835369,616.38146794)(140.28998899,617.04501001)
\curveto(141.63164819,617.70857565)(142.75688446,618.18095519)(143.61556566,618.41784023)
\curveto(144.4757908,618.6561158)(145.06475708,618.65472527)(145.38711951,618.41784023)
\curveto(145.71107466,618.18239286)(145.76529075,617.70857565)(145.60469908,617.04501001)
\curveto(145.44408423,616.38146794)(145.06866138,615.52813031)(144.57403785,614.56948949)
\curveto(144.07943944,613.61087223)(143.46563939,612.54697536)(142.85186373,611.48305492)
}
}
{
\newrgbcolor{curcolor}{0 0 0}
\pscustom[linewidth=0.90964309,linecolor=curcolor]
{
\newpath
\moveto(130.45089265,611.48305492)
\curveto(132.24572532,612.54697536)(134.03898893,613.6094817)(135.7162121,614.56948949)
\curveto(137.39186259,615.52813031)(138.94835369,616.38146794)(140.28998899,617.04501001)
\curveto(141.63164819,617.70857565)(142.75688446,618.18095519)(143.61556566,618.41784023)
\curveto(144.4757908,618.6561158)(145.06475708,618.65472527)(145.38711951,618.41784023)
\curveto(145.71107466,618.18239286)(145.76529075,617.70857565)(145.60469908,617.04501001)
\curveto(145.44408423,616.38146794)(145.06866138,615.52813031)(144.57403785,614.56948949)
\curveto(144.07943944,613.61087223)(143.46563939,612.54697536)(142.85186373,611.48305492)
}
}
{
\newrgbcolor{curcolor}{0.3882353 0.3882353 0.39607844}
\pscustom[linestyle=none,fillstyle=solid,fillcolor=curcolor,opacity=0.435294]
{
\newpath
\moveto(130.45200601,611.48237144)
\curveto(131.04252903,613.29241729)(131.63305109,615.10010631)(132.22357314,616.73338988)
\curveto(132.81409037,618.36431661)(133.40462449,619.81612422)(133.99513448,620.94504605)
\curveto(134.5856686,622.07396788)(135.17617859,622.87764709)(135.76671271,623.28066511)
\curveto(136.3572227,623.68603997)(136.94775683,623.68368314)(137.53826681,623.28066511)
\curveto(138.12880094,622.88000392)(138.71931092,622.07396788)(139.30984505,620.94504605)
\curveto(139.90035503,619.81612422)(140.49088916,618.36431661)(141.08139914,616.73338988)
\curveto(141.67193327,615.10246314)(142.26244325,613.29241729)(142.85297738,611.48237144)
}
}
{
\newrgbcolor{curcolor}{0 0 0}
\pscustom[linewidth=0.90964309,linecolor=curcolor]
{
\newpath
\moveto(130.45200601,611.48237144)
\curveto(131.04252903,613.29241729)(131.63305109,615.10010631)(132.22357314,616.73338988)
\curveto(132.81409037,618.36431661)(133.40462449,619.81612422)(133.99513448,620.94504605)
\curveto(134.5856686,622.07396788)(135.17617859,622.87764709)(135.76671271,623.28066511)
\curveto(136.3572227,623.68603997)(136.94775683,623.68368314)(137.53826681,623.28066511)
\curveto(138.12880094,622.88000392)(138.71931092,622.07396788)(139.30984505,620.94504605)
\curveto(139.90035503,619.81612422)(140.49088916,618.36431661)(141.08139914,616.73338988)
\curveto(141.67193327,615.10246314)(142.26244325,613.29241729)(142.85297738,611.48237144)
}
}
{
\newrgbcolor{curcolor}{0.3882353 0.3882353 0.39607844}
\pscustom[linestyle=none,fillstyle=solid,fillcolor=curcolor,opacity=0.435294]
{
\newpath
\moveto(142.85186855,611.48571814)
\curveto(143.44239158,609.67567228)(144.03291363,607.86798326)(144.62343569,606.2346997)
\curveto(145.21395291,604.60377296)(145.80448728,603.15196535)(146.39499726,602.02304352)
\curveto(146.98553115,600.8941217)(147.57604137,600.09044248)(148.1665755,599.68742446)
\curveto(148.75708548,599.28204961)(149.34761985,599.28440644)(149.93812935,599.68742446)
\curveto(150.52866348,600.08808565)(151.1191737,600.8941217)(151.70970759,602.02304352)
\curveto(152.30021806,603.15196535)(152.89075194,604.60377296)(153.48126168,606.2346997)
\curveto(154.07179654,607.86562643)(154.66231497,609.67567228)(155.2528491,611.48571814)
}
}
{
\newrgbcolor{curcolor}{0 0 0}
\pscustom[linewidth=0.90964309,linecolor=curcolor]
{
\newpath
\moveto(142.85186855,611.48571814)
\curveto(143.44239158,609.67567228)(144.03291363,607.86798326)(144.62343569,606.2346997)
\curveto(145.21395291,604.60377296)(145.80448728,603.15196535)(146.39499726,602.02304352)
\curveto(146.98553115,600.8941217)(147.57604137,600.09044248)(148.1665755,599.68742446)
\curveto(148.75708548,599.28204961)(149.34761985,599.28440644)(149.93812935,599.68742446)
\curveto(150.52866348,600.08808565)(151.1191737,600.8941217)(151.70970759,602.02304352)
\curveto(152.30021806,603.15196535)(152.89075194,604.60377296)(153.48126168,606.2346997)
\curveto(154.07179654,607.86562643)(154.66231497,609.67567228)(155.2528491,611.48571814)
}
}
{
\newrgbcolor{curcolor}{0.15686275 0.35294119 0.47450981}
\pscustom[linestyle=none,fillstyle=solid,fillcolor=curcolor]
{
\newpath
\moveto(142.85298221,611.48503466)
\curveto(142.23919567,610.42111421)(141.62697603,609.35860787)(141.13079673,608.39860009)
\curveto(140.63618069,607.43995927)(140.26075857,606.58662163)(140.10014299,605.92307956)
\curveto(139.93955229,605.25951392)(139.99533622,604.78713438)(140.31772256,604.55024934)
\curveto(140.63851786,604.31197377)(141.23061984,604.3133643)(142.08927665,604.55024934)
\curveto(142.94638951,604.78569671)(144.07319387,605.25951392)(145.41485332,605.92307956)
\curveto(146.75648886,606.58662163)(148.31297948,607.43995927)(149.98862248,608.39860009)
\curveto(151.66429084,609.35721734)(153.45911834,610.42111421)(155.25396226,611.48503466)
}
}
{
\newrgbcolor{curcolor}{0 0 0}
\pscustom[linewidth=0.90964309,linecolor=curcolor]
{
\newpath
\moveto(142.85298221,611.48503466)
\curveto(142.23919567,610.42111421)(141.62697603,609.35860787)(141.13079673,608.39860009)
\curveto(140.63618069,607.43995927)(140.26075857,606.58662163)(140.10014299,605.92307956)
\curveto(139.93955229,605.25951392)(139.99533622,604.78713438)(140.31772256,604.55024934)
\curveto(140.63851786,604.31197377)(141.23061984,604.3133643)(142.08927665,604.55024934)
\curveto(142.94638951,604.78569671)(144.07319387,605.25951392)(145.41485332,605.92307956)
\curveto(146.75648886,606.58662163)(148.31297948,607.43995927)(149.98862248,608.39860009)
\curveto(151.66429084,609.35721734)(153.45911834,610.42111421)(155.25396226,611.48503466)
}
}
{
\newrgbcolor{curcolor}{0.15686275 0.35294119 0.47450981}
\pscustom[linestyle=none,fillstyle=solid,fillcolor=curcolor]
{
\newpath
\moveto(155.24725044,611.4830587)
\curveto(157.04208311,612.54697914)(158.83534672,613.60948548)(160.51256989,614.56949327)
\curveto(162.18822038,615.52813409)(163.74471149,616.38147172)(165.08634679,617.04501379)
\curveto(166.42800599,617.70857943)(167.55324226,618.18095897)(168.41192346,618.41784401)
\curveto(169.2721486,618.65611958)(169.86111488,618.65472905)(170.18347731,618.41784401)
\curveto(170.50743246,618.18239664)(170.56164854,617.70857943)(170.40105687,617.04501379)
\curveto(170.24044202,616.38147172)(169.86501917,615.52813409)(169.37039565,614.56949327)
\curveto(168.87579724,613.61087601)(168.26199719,612.54697914)(167.64822152,611.4830587)
}
}
{
\newrgbcolor{curcolor}{0 0 0}
\pscustom[linewidth=0.90964309,linecolor=curcolor]
{
\newpath
\moveto(155.24725044,611.4830587)
\curveto(157.04208311,612.54697914)(158.83534672,613.60948548)(160.51256989,614.56949327)
\curveto(162.18822038,615.52813409)(163.74471149,616.38147172)(165.08634679,617.04501379)
\curveto(166.42800599,617.70857943)(167.55324226,618.18095897)(168.41192346,618.41784401)
\curveto(169.2721486,618.65611958)(169.86111488,618.65472905)(170.18347731,618.41784401)
\curveto(170.50743246,618.18239664)(170.56164854,617.70857943)(170.40105687,617.04501379)
\curveto(170.24044202,616.38147172)(169.86501917,615.52813409)(169.37039565,614.56949327)
\curveto(168.87579724,613.61087601)(168.26199719,612.54697914)(167.64822152,611.4830587)
}
}
{
\newrgbcolor{curcolor}{0.3882353 0.3882353 0.39607844}
\pscustom[linestyle=none,fillstyle=solid,fillcolor=curcolor,opacity=0.435294]
{
\newpath
\moveto(155.2483638,611.48237522)
\curveto(155.83888683,613.29242107)(156.42940888,615.10011009)(157.01993094,616.73339366)
\curveto(157.61044816,618.36432039)(158.20098229,619.816128)(158.79149227,620.94504983)
\curveto(159.3820264,622.07397166)(159.97253638,622.87765087)(160.56307051,623.28066889)
\curveto(161.15358049,623.68604375)(161.74411462,623.68368692)(162.3346246,623.28066889)
\curveto(162.92515873,622.8800077)(163.51566871,622.07397166)(164.10620284,620.94504983)
\curveto(164.69671282,619.816128)(165.28724695,618.36432039)(165.87775693,616.73339366)
\curveto(166.46829106,615.10246692)(167.05880104,613.29242107)(167.64933517,611.48237522)
}
}
{
\newrgbcolor{curcolor}{0 0 0}
\pscustom[linewidth=0.90964309,linecolor=curcolor]
{
\newpath
\moveto(155.2483638,611.48237522)
\curveto(155.83888683,613.29242107)(156.42940888,615.10011009)(157.01993094,616.73339366)
\curveto(157.61044816,618.36432039)(158.20098229,619.816128)(158.79149227,620.94504983)
\curveto(159.3820264,622.07397166)(159.97253638,622.87765087)(160.56307051,623.28066889)
\curveto(161.15358049,623.68604375)(161.74411462,623.68368692)(162.3346246,623.28066889)
\curveto(162.92515873,622.8800077)(163.51566871,622.07397166)(164.10620284,620.94504983)
\curveto(164.69671282,619.816128)(165.28724695,618.36432039)(165.87775693,616.73339366)
\curveto(166.46829106,615.10246692)(167.05880104,613.29242107)(167.64933517,611.48237522)
}
}
{
\newrgbcolor{curcolor}{0.3882353 0.3882353 0.39607844}
\pscustom[linestyle=none,fillstyle=solid,fillcolor=curcolor,opacity=0.435294]
{
\newpath
\moveto(167.64822635,611.48572192)
\curveto(168.23874937,609.67567606)(168.82927143,607.86798704)(169.41979348,606.23470348)
\curveto(170.01031071,604.60377674)(170.60084508,603.15196913)(171.19135506,602.0230473)
\curveto(171.78188895,600.89412548)(172.37239917,600.09044626)(172.9629333,599.68742824)
\curveto(173.55344328,599.28205339)(174.14397765,599.28441022)(174.73448715,599.68742824)
\curveto(175.32502128,600.08808943)(175.9155315,600.89412548)(176.50606539,602.0230473)
\curveto(177.09657585,603.15196913)(177.68710974,604.60377674)(178.27761948,606.23470348)
\curveto(178.86815433,607.86563021)(179.45867276,609.67567606)(180.04920689,611.48572192)
}
}
{
\newrgbcolor{curcolor}{0 0 0}
\pscustom[linewidth=0.90964309,linecolor=curcolor]
{
\newpath
\moveto(167.64822635,611.48572192)
\curveto(168.23874937,609.67567606)(168.82927143,607.86798704)(169.41979348,606.23470348)
\curveto(170.01031071,604.60377674)(170.60084508,603.15196913)(171.19135506,602.0230473)
\curveto(171.78188895,600.89412548)(172.37239917,600.09044626)(172.9629333,599.68742824)
\curveto(173.55344328,599.28205339)(174.14397765,599.28441022)(174.73448715,599.68742824)
\curveto(175.32502128,600.08808943)(175.9155315,600.89412548)(176.50606539,602.0230473)
\curveto(177.09657585,603.15196913)(177.68710974,604.60377674)(178.27761948,606.23470348)
\curveto(178.86815433,607.86563021)(179.45867276,609.67567606)(180.04920689,611.48572192)
}
}
{
\newrgbcolor{curcolor}{0.15686275 0.35294119 0.47450981}
\pscustom[linestyle=none,fillstyle=solid,fillcolor=curcolor]
{
\newpath
\moveto(167.64934,611.48503844)
\curveto(167.03555347,610.42111799)(166.42333383,609.35861165)(165.92715452,608.39860387)
\curveto(165.43253848,607.43996305)(165.05711636,606.58662541)(164.89650079,605.92308334)
\curveto(164.73591008,605.2595177)(164.79169402,604.78713816)(165.11408035,604.55025312)
\curveto(165.43487565,604.31197755)(166.02697763,604.31336808)(166.88563444,604.55025312)
\curveto(167.74274731,604.78570049)(168.86955167,605.2595177)(170.21121111,605.92308334)
\curveto(171.55284665,606.58662541)(173.10933727,607.43996305)(174.78498028,608.39860387)
\curveto(176.46064864,609.35722112)(178.25547614,610.42111799)(180.05032006,611.48503844)
}
}
{
\newrgbcolor{curcolor}{0 0 0}
\pscustom[linewidth=0.90964309,linecolor=curcolor]
{
\newpath
\moveto(167.64934,611.48503844)
\curveto(167.03555347,610.42111799)(166.42333383,609.35861165)(165.92715452,608.39860387)
\curveto(165.43253848,607.43996305)(165.05711636,606.58662541)(164.89650079,605.92308334)
\curveto(164.73591008,605.2595177)(164.79169402,604.78713816)(165.11408035,604.55025312)
\curveto(165.43487565,604.31197755)(166.02697763,604.31336808)(166.88563444,604.55025312)
\curveto(167.74274731,604.78570049)(168.86955167,605.2595177)(170.21121111,605.92308334)
\curveto(171.55284665,606.58662541)(173.10933727,607.43996305)(174.78498028,608.39860387)
\curveto(176.46064864,609.35722112)(178.25547614,610.42111799)(180.05032006,611.48503844)
}
}
{
\newrgbcolor{curcolor}{0 0 0}
\pscustom[linewidth=0.99999871,linecolor=curcolor]
{
\newpath
\moveto(180.86020913,585.00737008)
\lineto(180.86020913,553.70275276)
\lineto(199.84082646,553.70275276)
}
}
{
\newrgbcolor{curcolor}{1 1 1}
\pscustom[linestyle=none,fillstyle=solid,fillcolor=curcolor]
{
\newpath
\moveto(180.99538728,587.42678342)
\curveto(181.84837225,587.42678342)(182.53985319,586.39067649)(182.53985319,585.11257382)
\curveto(182.53985319,583.83447115)(181.84837225,582.79836422)(180.99538728,582.79836422)
\curveto(180.14240231,582.79836422)(179.45092137,583.83447115)(179.45092137,585.11257382)
\curveto(179.45092137,586.39067649)(180.14240231,587.42678342)(180.99538728,587.42678342)
}
}
{
\newrgbcolor{curcolor}{0 0 0.01176471}
\pscustom[linewidth=1.13385831,linecolor=curcolor]
{
\newpath
\moveto(180.99538728,587.42678342)
\curveto(181.84837225,587.42678342)(182.53985319,586.39067649)(182.53985319,585.11257382)
\curveto(182.53985319,583.83447115)(181.84837225,582.79836422)(180.99538728,582.79836422)
\curveto(180.14240231,582.79836422)(179.45092137,583.83447115)(179.45092137,585.11257382)
\curveto(179.45092137,586.39067649)(180.14240231,587.42678342)(180.99538728,587.42678342)
}
}
{
\newrgbcolor{curcolor}{0.78431374 0.78431374 0.78431374}
\pscustom[linestyle=none,fillstyle=solid,fillcolor=curcolor]
{
\newpath
\moveto(202.3186268,553.75955416)
\curveto(202.3186268,554.89139421)(200.95028341,555.45801667)(200.15004866,554.65778191)
\curveto(199.3498139,553.85754716)(199.91643636,552.48920377)(201.04827641,552.48920377)
\curveto(202.18011645,552.48920377)(202.74673891,553.85754716)(201.94650415,554.65778191)
\curveto(201.1462694,555.45801667)(199.77792601,554.89139421)(199.77792601,553.75955416)
\curveto(199.77792601,552.62771412)(201.1462694,552.06109166)(201.94650415,552.86132642)
\curveto(202.74673891,553.66156117)(202.18011645,555.02990456)(201.04827641,555.02990456)
\curveto(199.91643636,555.02990456)(199.3498139,553.66156117)(200.15004866,552.86132642)
\curveto(200.95028341,552.06109166)(202.3186268,552.62771412)(202.3186268,553.75955416)
\closepath
}
}
{
\newrgbcolor{curcolor}{0 0 0}
\pscustom[linewidth=1.13385831,linecolor=curcolor]
{
\newpath
\moveto(202.3186268,553.75955416)
\curveto(202.3186268,554.89139421)(200.95028341,555.45801667)(200.15004866,554.65778191)
\curveto(199.3498139,553.85754716)(199.91643636,552.48920377)(201.04827641,552.48920377)
\curveto(202.18011645,552.48920377)(202.74673891,553.85754716)(201.94650415,554.65778191)
\curveto(201.1462694,555.45801667)(199.77792601,554.89139421)(199.77792601,553.75955416)
\curveto(199.77792601,552.62771412)(201.1462694,552.06109166)(201.94650415,552.86132642)
\curveto(202.74673891,553.66156117)(202.18011645,555.02990456)(201.04827641,555.02990456)
\curveto(199.91643636,555.02990456)(199.3498139,553.66156117)(200.15004866,552.86132642)
\curveto(200.95028341,552.06109166)(202.3186268,552.62771412)(202.3186268,553.75955416)
\closepath
}
}
{
\newrgbcolor{curcolor}{0.78431374 0.78431374 0.78431374}
\pscustom[linestyle=none,fillstyle=solid,fillcolor=curcolor]
{
\newpath
\moveto(214.92713523,553.75955416)
\curveto(214.92713523,553.05795895)(215.49589052,552.48920366)(216.19748574,552.48920366)
\curveto(216.89908095,552.48920366)(217.46783624,553.05795895)(217.46783624,553.75955416)
\curveto(217.46783624,554.46114938)(216.89908095,555.02990467)(216.19748574,555.02990467)
\curveto(215.49589052,555.02990467)(214.92713523,554.46114938)(214.92713523,553.75955416)
}
}
{
\newrgbcolor{curcolor}{0 0 0}
\pscustom[linewidth=1.13385831,linecolor=curcolor]
{
\newpath
\moveto(214.92713523,553.75955416)
\curveto(214.92713523,553.05795895)(215.49589052,552.48920366)(216.19748574,552.48920366)
\curveto(216.89908095,552.48920366)(217.46783624,553.05795895)(217.46783624,553.75955416)
\curveto(217.46783624,554.46114938)(216.89908095,555.02990467)(216.19748574,555.02990467)
\curveto(215.49589052,555.02990467)(214.92713523,554.46114938)(214.92713523,553.75955416)
}
}
{
\newrgbcolor{curcolor}{0 0 0}
\pscustom[linestyle=none,fillstyle=solid,fillcolor=curcolor]
{
\newpath
\moveto(208.01606993,541.2253937)
\lineto(204.30513003,550.94545206)
\lineto(205.67882884,550.94545206)
\lineto(208.75825791,542.76185303)
\lineto(211.8441974,550.94545206)
\lineto(213.21138578,550.94545206)
\lineto(209.50695631,541.2253937)
\closepath
}
}
{
\newrgbcolor{curcolor}{0 0 0}
\pscustom[linewidth=0.99999871,linecolor=curcolor]
{
\newpath
\moveto(180.70112882,644.12352756)
\lineto(180.70112882,666.90678425)
}
}
{
\newrgbcolor{curcolor}{0 0 0}
\pscustom[linewidth=0.99999871,linecolor=curcolor]
{
\newpath
\moveto(180.76614803,662.06630551)
\lineto(237.31197732,662.06630551)
}
}
{
\newrgbcolor{curcolor}{0 0 0}
\pscustom[linewidth=0.99999871,linecolor=curcolor]
{
\newpath
\moveto(237.67339843,644.12352756)
\lineto(237.67339843,666.90678425)
}
}
{
\newrgbcolor{curcolor}{0 0 0}
\pscustom[linestyle=none,fillstyle=solid,fillcolor=curcolor]
{
\newpath
\moveto(206.75373421,675.23487913)
\lineto(208.0688355,675.23487913)
\lineto(208.0688355,666.62161677)
\lineto(212.80189803,666.62161677)
\lineto(212.80189803,665.51484836)
\lineto(206.75373421,665.51484836)
\closepath
}
}
{
\newrgbcolor{curcolor}{0.15686275 0.35294119 0.47450981}
\pscustom[linestyle=none,fillstyle=solid,fillcolor=curcolor]
{
\newpath
\moveto(105.67327753,611.4830569)
\curveto(107.46811017,612.54697732)(109.26137375,613.60948365)(110.93859689,614.56949142)
\curveto(112.61424736,615.52813222)(114.17073844,616.38146984)(115.51237372,617.0450119)
\curveto(116.8540329,617.70857753)(117.97926915,618.18095707)(118.83795033,618.4178421)
\curveto(119.69817546,618.65611767)(120.28714173,618.65472714)(120.60950416,618.4178421)
\curveto(120.9334593,618.18239473)(120.98767539,617.70857753)(120.82708372,617.0450119)
\curveto(120.66646887,616.38146984)(120.29104603,615.52813222)(119.79642251,614.56949142)
\curveto(119.30182411,613.61087418)(118.68802407,612.54697732)(118.07424841,611.4830569)
}
}
{
\newrgbcolor{curcolor}{0 0 0}
\pscustom[linewidth=0.90964308,linecolor=curcolor]
{
\newpath
\moveto(105.67327753,611.4830569)
\curveto(107.46811017,612.54697732)(109.26137375,613.60948365)(110.93859689,614.56949142)
\curveto(112.61424736,615.52813222)(114.17073844,616.38146984)(115.51237372,617.0450119)
\curveto(116.8540329,617.70857753)(117.97926915,618.18095707)(118.83795033,618.4178421)
\curveto(119.69817546,618.65611767)(120.28714173,618.65472714)(120.60950416,618.4178421)
\curveto(120.9334593,618.18239473)(120.98767539,617.70857753)(120.82708372,617.0450119)
\curveto(120.66646887,616.38146984)(120.29104603,615.52813222)(119.79642251,614.56949142)
\curveto(119.30182411,613.61087418)(118.68802407,612.54697732)(118.07424841,611.4830569)
}
}
{
\newrgbcolor{curcolor}{0.3882353 0.3882353 0.39607844}
\pscustom[linestyle=none,fillstyle=solid,fillcolor=curcolor,opacity=0.435294]
{
\newpath
\moveto(105.67439089,611.48237342)
\curveto(106.2649139,613.29241924)(106.85543595,615.10010824)(107.44595799,616.73339177)
\curveto(108.03647521,618.36431848)(108.62700933,619.81612607)(109.2175193,620.94504788)
\curveto(109.80805342,622.07396969)(110.39856339,622.87764889)(110.98909751,623.28066691)
\curveto(111.57960748,623.68604175)(112.1701416,623.68368492)(112.76065157,623.28066691)
\curveto(113.35118569,622.88000572)(113.94169567,622.07396969)(114.53222979,620.94504788)
\curveto(115.12273976,619.81612607)(115.71327388,618.36431848)(116.30378385,616.73339177)
\curveto(116.89431797,615.10246507)(117.48482794,613.29241924)(118.07536206,611.48237342)
}
}
{
\newrgbcolor{curcolor}{0 0 0}
\pscustom[linewidth=0.90964308,linecolor=curcolor]
{
\newpath
\moveto(105.67439089,611.48237342)
\curveto(106.2649139,613.29241924)(106.85543595,615.10010824)(107.44595799,616.73339177)
\curveto(108.03647521,618.36431848)(108.62700933,619.81612607)(109.2175193,620.94504788)
\curveto(109.80805342,622.07396969)(110.39856339,622.87764889)(110.98909751,623.28066691)
\curveto(111.57960748,623.68604175)(112.1701416,623.68368492)(112.76065157,623.28066691)
\curveto(113.35118569,622.88000572)(113.94169567,622.07396969)(114.53222979,620.94504788)
\curveto(115.12273976,619.81612607)(115.71327388,618.36431848)(116.30378385,616.73339177)
\curveto(116.89431797,615.10246507)(117.48482794,613.29241924)(118.07536206,611.48237342)
}
}
{
\newrgbcolor{curcolor}{0.3882353 0.3882353 0.39607844}
\pscustom[linestyle=none,fillstyle=solid,fillcolor=curcolor,opacity=0.435294]
{
\newpath
\moveto(118.07425324,611.48572012)
\curveto(118.66477625,609.67567429)(119.2552983,607.8679853)(119.84582034,606.23470176)
\curveto(120.43633756,604.60377506)(121.02687192,603.15196747)(121.61738189,602.02304566)
\curveto(122.20791577,600.89412385)(122.79842598,600.09044465)(123.3889601,599.68742663)
\curveto(123.97947008,599.28205179)(124.57000444,599.28440862)(125.16051393,599.68742663)
\curveto(125.75104805,600.08808782)(126.34155826,600.89412385)(126.93209214,602.02304566)
\curveto(127.52260259,603.15196747)(128.11313647,604.60377506)(128.7036462,606.23470176)
\curveto(129.29418104,607.86562847)(129.88469947,609.67567429)(130.47523359,611.48572012)
}
}
{
\newrgbcolor{curcolor}{0 0 0}
\pscustom[linewidth=0.90964308,linecolor=curcolor]
{
\newpath
\moveto(118.07425324,611.48572012)
\curveto(118.66477625,609.67567429)(119.2552983,607.8679853)(119.84582034,606.23470176)
\curveto(120.43633756,604.60377506)(121.02687192,603.15196747)(121.61738189,602.02304566)
\curveto(122.20791577,600.89412385)(122.79842598,600.09044465)(123.3889601,599.68742663)
\curveto(123.97947008,599.28205179)(124.57000444,599.28440862)(125.16051393,599.68742663)
\curveto(125.75104805,600.08808782)(126.34155826,600.89412385)(126.93209214,602.02304566)
\curveto(127.52260259,603.15196747)(128.11313647,604.60377506)(128.7036462,606.23470176)
\curveto(129.29418104,607.86562847)(129.88469947,609.67567429)(130.47523359,611.48572012)
}
}
{
\newrgbcolor{curcolor}{0.15686275 0.35294119 0.47450981}
\pscustom[linestyle=none,fillstyle=solid,fillcolor=curcolor]
{
\newpath
\moveto(118.07536689,611.48503664)
\curveto(117.46158037,610.42111621)(116.84936073,609.35860989)(116.35318144,608.39860212)
\curveto(115.85856541,607.43996131)(115.48314329,606.58662369)(115.32252772,605.92308163)
\curveto(115.16193702,605.25951601)(115.21772095,604.78713647)(115.54010728,604.55025144)
\curveto(115.86090258,604.31197587)(116.45300455,604.3133664)(117.31166134,604.55025144)
\curveto(118.16877419,604.7856988)(119.29557854,605.25951601)(120.63723796,605.92308163)
\curveto(121.97887348,606.58662369)(123.53536408,607.43996131)(125.21100706,608.39860212)
\curveto(126.88667539,609.35721936)(128.68150286,610.42111621)(130.47634676,611.48503664)
}
}
{
\newrgbcolor{curcolor}{0 0 0}
\pscustom[linewidth=0.90964308,linecolor=curcolor]
{
\newpath
\moveto(118.07536689,611.48503664)
\curveto(117.46158037,610.42111621)(116.84936073,609.35860989)(116.35318144,608.39860212)
\curveto(115.85856541,607.43996131)(115.48314329,606.58662369)(115.32252772,605.92308163)
\curveto(115.16193702,605.25951601)(115.21772095,604.78713647)(115.54010728,604.55025144)
\curveto(115.86090258,604.31197587)(116.45300455,604.3133664)(117.31166134,604.55025144)
\curveto(118.16877419,604.7856988)(119.29557854,605.25951601)(120.63723796,605.92308163)
\curveto(121.97887348,606.58662369)(123.53536408,607.43996131)(125.21100706,608.39860212)
\curveto(126.88667539,609.35721936)(128.68150286,610.42111621)(130.47634676,611.48503664)
}
}
\end{pspicture}

\caption{Schematic view of light passing through an \ac{eom}. The electrodes are positioned on the front and back side of the modulator, the same faces the light enters and exits.}
\label{fig:topView}
}
\end{figure}

\subsection{Theory}
	\subsection{Chopping in the experiment}
	\subsection{Bergmann pockels cell driver}

\section{Acousto-optical modulators}
	\subsection{Operation}
	\subsection{Usage in the experiment}

