\chapter{Introduction}

\newcommand{\docite}{\textcolor{red}{~█}}

Quantum mechanics provided a new way of thinking about physics and experimentalists and theorists have worked together in the last century, to validate this understanding. Nowadays, within the realm of quantum research, atoms are being controlled in an isolated environment, that have allowed to set e.g.\ new high-precision time standards~\cite{Brewer2019} and resolve new types of matter~\cite{Anderson1995, Davis1995}. The scope of these studies has exponentially expanded over the last 30 years, as laser-cooling, trapping and detection techniques have gained momentum. Within this realm, the field of quantum simulation is of particular interest to this thesis, where microscopic effects found in solid state physics are modelled by arranging and interacting atoms on a macroscopic scale. Forming the structures necessary to measure effects in this regime, such as Ising~\cite{Schauß2015} or Hubbard~\cite{Boll2016} models, is achieved by confining atoms to periodic potential wells in optical lattices.

Advancements in detection of atoms have lead to quantum gas microscopes~\cite{Bakr2009, Sherson2010}, where single-site resolutions in these lattices is possible and are thus a perfect fit for use in quantum simulators. This has allowed to measure e.g.\ superfluid to Mott insulator transitions~\cite{Sherson2010} or antiferromagnetic correlations~\cite{Boll2016}. However, control over these systems was limited in the past, as optical lattices have a fixed geometry and cycle times were long due to the necessity of cooling atoms to ultra-cold temperatures. More recently however, optical tweezers have emerged as an alternative~\cite{Barredo2016, Endres2016, Norcia2018}, where each lattice site is generated from an individual laser beam. This allows high control and tunability not only between lattice sites, but also poses the ability to generate arbitrary geometries. As these traps can be made sufficiently deep, this removes the requirement of ultra-cooling atoms, resulting in fast cycle times.

To study manybody effects in these optical tweezers, Rydberg atoms can be employed. They offer the ability of long-range interactions on a micrometer scale~\cite{Urban2009} and therefore across multiple tweezers. This forms the building blocks for the experiment this thesis is a part of. Hereby, single Potassium-39 atoms are trapped in optical tweezers and excited to Rydberg levels where e.g.\ blockade effects are measured~\cite{Hirthe2018}. The tweezers are generated through \acp{slm}~\cite{Osterholz2020}, where arbitrary intensity patterns are formed and projected onto the atomic cloud.

This work presents three new setups towards this experiment. In Chapter~\ref{ch:chopping}, loading of atoms from the cloud into tweezers is discussed. The chapter places focus on the theory behind \acp{eom} which is followed by the characterization of two \acp{eom}, that will be used to improve the loading stage of the experiment and allow deploying more tweezers in the future.
Chapter~\ref{sec:sorting} then works towards enabling unity filling in a grid of atoms. For this, a new way of generating the tweezers is introduced in the form of \acp{aod}, which can move atoms into defects in the lattice. Therefore the theory behind the deflectors is explained and is followed by comparing two algorithms of achieving this sorting of atoms. The chapter is then concluded by describing the driver used for the dynamic tweezers and ways to work around its limitations.
Then Chapter~\ref{ch:spin_resolved} presents a spin-selective imaging approach by applying dynamic tweezers that are sensitive to only one spin species using the same \acp{aod} as above. The spin-sensitive laser used in this setup was built during this thesis and is concluded with the characterization of this laser.
