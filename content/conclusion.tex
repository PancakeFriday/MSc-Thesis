\chapter{Conclusion and Outlook}

This thesis discussed setups related to optical tweezer arrays in a Potassium-39 experiment. It was shown, how using a Pockels cell improves a current setup for loading atoms into \ac{slm} tweezers. With the new implementation, the duty cycle of the dipole traps during the chopping sequence can be increased, effectively improving the power available for every single tweezer. This also opens up the possibility of increasing the number of tweezer, as the trap depth available currently can then be distributed across a larger area. The discussion of sorting atoms followed, which solves the problem of only having $50\%$ of sites occupied due to parametric heating. The solution requires moving atoms into unoccupied sites, which can be implemented using a pathfinding algorithm. As an alternative, a new algorithm was proposed, that uses parallelization and a digitizer card to move multiple atoms at the same time. The chapter was concluded by comparing the efficiency of the algorithms and by highlighting the requirements when using the digitizer card. The third setup then discussed an approach of doing spin-selective imaging, which will be useful in experiments involving Rydberg dressed states, as they are mostly ground states, but of unknown spin. For this, the spin states are spatially separated by using spin selective light shifts induced by a frequency stabilized laser. The laser system was built and characterized, and a stable cavity with a feedback loop to the laser for frequency stabilization was described. It is then coupled into the same crossed \acp{aod} used for the sorting, allowing to move atoms (and therefore the spin states) apart.

Shown were ways of improving the current state of the experimental setup. Moreover, there are applications beyond the scope of the current system in place. As was shown, the \acp{eom} presented here also have faster repetition rates than the \acp{aom} used currently. Apart from loading, it is then possible to drive faster pi-pulses in the experiment, or start doing Floquet engineering~\cite{Oka2019}, which requires periodic driving of atoms. For sorting, being able to fill every site in an atomic array opens the door to manybody Rydberg effects. This allows for example to model antiferromagnets in an Ising-like system~\cite{Lienhard2018} and measure correlations without the need of heavy post-selection of data. It was motivated, how spin-selective imaging reduces cycle times, which ultimately leads to being able to take more statistics. This is a requirement to measuring entanglement entropies, for example Rényi entropies~\cite{Wilde2014} in a quantum information application of the experiment. Conclusively, the setups that were built over the course of this thesis offer a way to explore manybody Rydberg effects with high cycle times and high fidelity.
