\chapter{Conclusion and Outlook}

This thesis discussed setups related to optical tweezer arrays in a Potassium-39 experiment. It was shown, how using a Pockels cell, a current setup for loading atoms into \ac{slm} tweezers can be improved. With the new implementation, the duty cycle of the dipole traps during the chopping sequence can be increased, effectively improving the power available for every single tweezer. This also opens up the possibility of increasing the number of tweezer, as the trap depth available currently can then be distributed across a larger area. The discussion of sorting atoms followed, which solves the problem of only having $50\%$ of sites occupied due to parametric heating. The solution requires moving atoms into unocuppied sites, which can be implemented using a pathfinding algorithm. As an alternative, a new algorithm was proposed, which uses parallelization and a digitizer card to move multiple atoms at the same time. The chapter was concluded by comparing the efficiency of the algorithms and by highlighting the requirements when using the digitizer card. The third setup then discussed an approach of doing spin-selective imaging, which will be useful in any experiment doing Rydberg dressing. For this, the spin states are spatially separated by using spin selective light shifts induced by lasers. The laser system was built and characterized, including a cavity for frequency stabilization. It can then be coupled into the same crossed \acp{aod} used for the sorting, allowing to move atoms (and therefore the spin states) apart.
